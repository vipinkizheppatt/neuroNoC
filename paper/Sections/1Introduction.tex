\section{Introduction}
The growing demand for automated complex intelligent systems leads to dramatic changes in the development and deployment of Artificial Neural Networks (ANN). The capabilities of ANN to map, model and classify nonlinear systems have allowed to incorporate it in various applications in such fields as science, engineering and economics [n]. The development and implementation of ANN has been done mostly in software. However, despite the benefits of software-based ANN such as high level of abstraction (i.e. no need to know the inner workings of ANN for designers), it has severe problems in real-time applications in terms of execution time in contrast to its hardware-based counterparts [e]. In order to address that issue, there have been proposed several hardware adaptations of ANN [f]. Indeed, even though the software implementation offers flexibility, the high-speed operation in real-time applications is only achievable with the hardware-based networks [g].


There has been proposed numerous hardware architectures of ANN, which can be classified into two groups: analog and digital systems. The latter is more popular, since it provides higher accuracy, noise immunity, better scalability, higher flexibility and compatibility. There are three types of digital implementations of hardware-based ANN: field programmable gate array (FPGA) based, digital signal processor (DSP) based and application specific integrated circuit (ASIC) based implementations.  The latter two types of architectures are less suitable for ANN implementation than the first type. DSP based architecture, for instance, is mainly sequential, as a consequence, it lacks to provide parallelism in ANN. ASIC based architectures, on the other hand, does not offer flexibility to reconfigure the ANN by the user [n]. It must be noted that ASIC and DSP can be designed to be highly parallel, however, it is expensive and complicated process to design [e]. FPGA is the best option to implement ANN, since it offers fully parallel and reconfigurable design capabilities.


Reconfigurability of FPGA allows the user to design application specific hardware architecture [g]. FPGA implementations of ANN with a large number of neurons is one of the challenges in hardware-based realizations, since ANN algorithms usually consists of vast amount of multiplication processes and requires higher precision [e]. Hence, on-chip learning is considered to be difficult and useless, as it causes a loss of efficiency in a hardware realization. As a result, off-chip learning is usually chosen when there is no necessity in dynamic learning. Also, the design changes can be made within a few hours, which significantly saves the time and cost of design production. Most of the hardware-based realizations are implemented in such a way that the structure of ANN can be altered through the reprogramming of FPGA [e].


In this paper, we propose the open-source reconfigurable hardware-based realization of ANN, which allows to alter the structure of ANN through the routing of packets in a system. ANN is implemented by the use of mesh topology, which consists of four-way switches with a programming element (PE) connected to it. Each switch with its PE represents a single neuron in ANN. Off-chip learning is chosen in our architecture due to the above-mentioned advantages over on-chip learning. The novelty of the proposed architecture is in the capability of changing the interconnections, weighs and biases on each neuron without explicitly reprogramming the FPGA, but by introducing the ANN configuration packet format, which can be given to the chip the same way as the usual input data packet.  

The main contributions of this work are:
\begin{itemize}
    \item Detailed architecture description of the FPGA-based implementation of ANN
    \item An open-source implementation of ANN targeting the Xillinx Virtex-7
    \item  Characterization of the proposed platform in the terms of resource utilization and performance.
\end{itemize}
 
The rest of the paper is organized as following: Section 2 provides the background, Section 3 discusses the background, Section 4 presents the detailed description of the proposed architecture, Section 5 provides the results f the resource utilization and performance analysis, and Section 6 concludes the paper.
